%%%%%%%%%%%%%%%%%%%%%%%%%%%%%%%%%%%%%%%%%%%%%%%%%%%%%%%%%%%%
%USE THIS FOR PREPARING DRAFT AND FOR WEBSITE

%READY FOR ELSEVIER PUBLICATIONS

%SINGLE COLUMN FORMAT
%\documentclass[final,authoryear,11pt]{elsarticle}
%\documentclass[5p,authoryear]{elsarticle} %TWO COLUMN FORMAT
\documentclass[authoryear]{elsarticle} %SINGLE COLUMN

\usepackage{fullpage}
\oddsidemargin  -0.3in
\evensidemargin -0.3in
\textwidth      7.1in
\headheight     0.0in
\topmargin      -0.5in
\textheight     9.5in

%\usepackage{endfloat}

\journal{Ocean Engineering}

\usepackage{enumitem}
%%%%%%%%%%%%%%%%%%%%%%%%%%%%%%%%%%%%%%%%%%%%%%%%%%%%%%%%%%%%
%USE THIS FOR SUBMISSION
%\documentclass[cmbright,fleqn,referee]{envauth}
%\received{00 Month 2012}
%\revised{00 Month 2012}
%\accepted{00 Month 2012}
%Also uncomment the title section below
%%%%%%%%%%%%%%%%%%%%%%%%%%%%%%%%%%%%%%%%%%%%%%%%%%%%%%%%%%%%=

%%%%%%%%%%%%%%%%%%%%%%%%%%%%%%%%%%%%%%%%%%%%%%%%%%%%%%%%%%%%
%Packages to use regardless of document class
%\usepackage{natbib}
\usepackage{upgreek}
\usepackage{mathtools}
\usepackage{rotating}
\usepackage{amsmath}
\usepackage{amssymb}
\usepackage{bbm}
\usepackage{subcaption}
\usepackage{graphicx}
\usepackage{url}
\usepackage[linesnumbered]{algorithm2e}
%\usepackage[section]{placeins}
\usepackage{placeins}
%\usepackage{listings}
%\usepackage[framed,numbered]{matlab-prettifier}
%\usepackage{endfloat}
\usepackage{xcolor}
\usepackage{multirow}
\usepackage{mathrsfs}
%%%%%%%%%%%%%%%%%%%%%%%%%%%%%%%%%%%%%%%%%%%%%%%%%%%%%%%%%%%%

%%%%%%%%%%%%%%%%%%%%%%%%%%%%%%%%%%%%%%%%%%%%%%%%%%%%%%%%%%%%
\definecolor{Ph}{rgb}{1,0,0}
%%%%%%%%%%%%%%%%%%%%%%%%%%%%%%%%%%%%%%%%%%%%%%%%%%%%%%%%%%%%

%%%%%%%%%%%%%%%%%%%%%%%%%%%%%%%%%%%%%%%%%%%%%%%%%%%%%%%%%%%%
%Notation
\def\bSig\mathbf{\Sigma}
\newcommand {\VS}{V\&S}
\newcommand {\tr}{\mbox{tr}}
%
\newcommand {\un}[1]{\boldsymbol{#1}}
%
\newcommand {\pbi}{\begin{itemize}}
\newcommand {\pei}{\end{itemize}}
\newcommand {\pii}{\item}
%
\newcommand {\pbc}{\begin{center}}
\newcommand {\pec}{\end{center}}
%
\newcommand {\pbe}{\begin{eqnarray*}}
\newcommand {\pee}{\end{eqnarray*}}
%
\newcommand {\pben}{\begin{eqnarray}}
\newcommand {\peen}{\end{eqnarray}}
%
\providecommand{\Pr}{\mathbb{Pr}} %real numbers
\let\hat=\widehat
\let\geq=\geqslant
\let\leq=\leqslant
%
\newcommand{\ph}[1]{\textcolor{red}{***[#1]}}
%
%\newcommand{\coloneqq}{\,\stackrel{{\rm \vartriangle}}{=}\,}
\newcommand {\pms}{\quad}
\newcommand {\simindep}{\,\stackrel{{\rm indep}}{\sim}\,}
\newcommand {\med}[1]{\underset{#1}{\mathrm{med}}}
\newcommand {\maxover}[1]{\underset{#1}{\mathrm{max} \text{ }}}
\newcommand {\minover}[1]{\underset{#1}{\mathrm{min} \text{ }}}
\newcommand {\meanover}[1]{\underset{#1}{\mathrm{mean} \text{ }}}
\newcommand {\argmax}[1]{\underset{#1}{\mathrm{argmax} \text{ }}}
\newcommand {\argmin}[1]{\underset{#1}{\mathrm{argmin} \text{ }}}
\newcommand {\bs}[1]{\boldsymbol{#1}}
%
%
\newcommand {\ed}[1]{\textcolor{black}{#1}}
%\newcommand {\ed}[1]{\textcolor{red}{#1}}
%

\providecommand{\Xb}{\boldsymbol{X}}
\providecommand{\xb}{\boldsymbol{x}}

\providecommand{\Xd}{\dot{X}}
\providecommand{\Yd}{\dot{Y}}
\providecommand{\zd}{\dot{z}}
\providecommand{\yd}{\dot{y}}
\providecommand{\xd}{\dot{x}}
\providecommand{\thetad}{\dot{\theta}}
\providecommand{\phid}{\dot{\phi}}
\providecommand{\nd}{{\dot{n}}}
\providecommand{\Dd}{{\dot{D}}}
%
\providecommand{\rhot}{\tilde{\rho}}
\providecommand{\sigmat}{\tilde{\sigma}}
\providecommand{\xit}{\tilde{\xi}}
\providecommand{\ut}{{\tilde{u}}}
\providecommand{\qt}{{\tilde{q}}}
\providecommand{\Qt}{{\tilde{Q}}}
\providecommand{\qb}{{\breve{q}}}
\providecommand{\Qb}{{\breve{Q}}}
%
\providecommand{\eps}{\epsilon}
\providecommand{\cvr}{{\theta,\phi}}
\providecommand{\cvrA}{{\theta \text{ and } \phi}}
\providecommand{\pr}{\sigma, \xi}
\providecommand{\prA}{\sigma \text{ and } \xi}
%\providecommand{\prP}{\sigma', \xi'}
\providecommand{\prP}{\alpha, \beta}
\providecommand{\prm}{\psi, \sigma, \xi}
\providecommand{\prmA}{\psi, \sigma  \text{ and } \xi}
%
\newcommand {\GP}{\mathrm{GP}}
\newcommand {\W}{\mathrm{Wbl}}
\newcommand {\TW}{\mathrm{TW}}
%
\usepackage {tikz}
\usetikzlibrary{arrows}
%
\setlength{\parindent}{0.2cm}

\usepackage{xspace} 

\providecommand{\RS}{\texttt{rsds}\xspace}
\providecommand{\WS}{\texttt{sfcWind}\xspace}
\providecommand{\WM}{\texttt{sfcWindmax}\xspace}
\providecommand{\TA}{\texttt{tas}\xspace}

\providecommand{\AC}{\texttt{ACCESS-CM2}\xspace}
\providecommand{\CA}{\texttt{CAMS-CSM1-0}\xspace}
\providecommand{\CE}{\texttt{CESM2}\xspace}
\providecommand{\EC}{\texttt{EC-Earth3}\xspace}
\providecommand{\MR}{\texttt{MRI-ESM2-0}\xspace}
\providecommand{\No}{\texttt{NorESM2-0-LL}\xspace}
\providecommand{\UK}{\texttt{UKESM1-0-LL}\xspace}

\providecommand{\SL}{\texttt{SSP126}\xspace}
\providecommand{\SM}{\texttt{SSP245}\xspace}
\providecommand{\SH}{\texttt{SSP585}\xspace}

\providecommand{\NA}{{North Atlantic}\xspace}
\providecommand{\CS}{{Celtic Sea}\xspace}

\begin{document}
\newcommand\Tstrut{\rule{0pt}{2.6ex}} 
% \begin{table}[!ht]
% 	\resizebox{\columnwidth}{!}{%
% 		\begin{tabular}{|c|l|l|l|l|}
% 			\hline
% 			\multicolumn{1}{|c|}{GCM\Tstrut} &
% 			\multicolumn{1}{|c|}{Variable\Tstrut} &
% 			\multicolumn{1}{|c|}{SSP126\Tstrut} &
% 			\multicolumn{1}{|c|}{SSP245\Tstrut} &
% 			\multicolumn{1}{|c|}{SSP585\Tstrut} \\ \hline  

% 			\multirow{2}{*}{\begin{tabular}{@{}c@{}}ACCESS-CM2 \\ (AC)\end{tabular}} &
% 			tas\Tstrut &
% 			r1i1p1f1, r2i1p1f1, r3i1p1f1, r4i1p1f1, r5i1p1f1 &
% 			r1i1p1f1, r2i1p1f1, r3i1p1f1, r4i1p1f1, r5i1p1f1 &
% 			r1i1p1f1, r2i1p1f1, r3i1p1f1, r4i1p1f1, r5i1p1f1 \\ \hline
			
% 			\multirow{1}{*}{\begin{tabular}{@{}c@{}}CESM2 \\ (CE)\end{tabular}} &
% 			tas\Tstrut &
% 			r4i1p1f1,   r10i1p1f1,r11i1p1f1 &
% 			r4i1p1f1,   r10i1p1f1,r11i1p1f1 &
% 			r4i1p1f1,   r10i1p1f1,r11i1p1f1 \\ \cline{2-5} 
			
% 			\multirow{1}{*}{\begin{tabular}{@{}c@{}}EC-Earth3 \\ (EC)\end{tabular}} &
% 			tas\Tstrut &
% 			r1i1p1f1,r4i1p1f1,   r11i1p1f1 &
% 			r1i1p1f1,r4i1p1f1,   r11i1p1f1 &
% 			r1i1p1f1,r4i1p1f1,   r11i1p1f1 \\ \hline 
			
% 			\multirow{1}{*}{\begin{tabular}{@{}c@{}}MRI-ESM2-0 \\ (MR)\end{tabular}} &
% 			tas\Tstrut &
% 			r1i1p1f1, r2i1p1f1,   r3i1p1f1, r4i1p1f1, r5i1p1f1 &
% 			r1i1p1f1, r2i1p1f1,   r3i1p1f1, r4i1p1f1, r5i1p1f1 &
% 			r1i1p1f1, r2i1p1f1,   r3i1p1f1, r4i1p1f1, r5i1p1f1 \\ \hline 
			
% 			\multirow{1}{*}{\begin{tabular}{@{}c@{}}UKESM1-0-LL \\ (UK)\end{tabular}} &
% 			tas\Tstrut &
% 			r1i1p1f2, r2i1p1f2, r3i1p1f2, r4i1p1f2, r8i1p1f2 &
% 			r1i1p1f2, r2i1p1f2, r3i1p1f2, r4i1p1f2, r8i1p1f2 &
% 			r1i1p1f2, r2i1p1f2, r3i1p1f2, r4i1p1f2, r8i1p1f2 \\ \hline
% 		\end{tabular}%
% 	}
% 	\caption{Summary of global coupled model (GCM) output considered. A total of 5 GCMs are used, listed in alphabetical order together with two-letter acronym (column 1). For each GCM, up to four climate variables are used, depending on their availability (column 2), for each of three climate scenarios (row 1, columns 3-5). A total of up to 5 ensemble members are considered for each combination of climate variable and scenario, again depending on availability (columns 3-5).}
% 	\label{Tbl:GcmDat}
% \end{table}

\begin{table}[!ht]
\resizebox{\columnwidth}{!}{%
    \begin{tabular}{|c|c|c|c|c|}
    \hline
    GCM & Variable & SSP126 & SSP245 & SSP585 \\ \hline

    ACCESS-CM2 (AC) & tas\Tstrut &
    r1i1p1f1, r2i1p1f1, r3i1p1f1, r4i1p1f1, r5i1p1f1 &
    r1i1p1f1, r2i1p1f1, r3i1p1f1, r4i1p1f1, r5i1p1f1 &
    r1i1p1f1, r2i1p1f1, r3i1p1f1, r4i1p1f1, r5i1p1f1 \\ \hline

    CESM2 (CE) & tas\Tstrut &
    r4i1p1f1, r10i1p1f1, r11i1p1f1 &
    r4i1p1f1, r10i1p1f1, r11i1p1f1 &
    r4i1p1f1, r10i1p1f1, r11i1p1f1 \\ \hline

    EC-Earth3 (EC) & tas\Tstrut &
    r1i1p1f1, r4i1p1f1, r11i1p1f1 &
    r1i1p1f1, r4i1p1f1, r11i1p1f1 &
    r1i1p1f1, r4i1p1f1, r11i1p1f1 \\ \hline

    MRI-ESM2-0 (MR) & tas\Tstrut &
    r1i1p1f1, r2i1p1f1, r3i1p1f1, r4i1p1f1, r5i1p1f1 &
    r1i1p1f1, r2i1p1f1, r3i1p1f1, r4i1p1f1, r5i1p1f1 &
    r1i1p1f1, r2i1p1f1, r3i1p1f1, r4i1p1f1, r5i1p1f1 \\ \hline

    UKESM1-0-LL (UK) & tas\Tstrut &
    r1i1p1f2, r2i1p1f2, r3i1p1f2, r4i1p1f2, r8i1p1f2 &
    r1i1p1f2, r2i1p1f2, r3i1p1f2, r4i1p1f2, r8i1p1f2 &
    r1i1p1f2, r2i1p1f2, r3i1p1f2, r4i1p1f2, r8i1p1f2 \\ \hline
    \end{tabular}%
}
\caption{Summary of global coupled model (GCM) output considered. A total of 5 GCMs are used, listed in alphabetical order together with two-letter acronym (column 1). For each GCM, the variable \TA was used, depending on availability (column 2), for each of three climate scenarios (row 1, columns 3-5). A total of up to 5 ensemble members are considered for each combination of climate variable and scenario, again depending on availability (columns 3-5).}
\label{Tbl:GcmDat}
\end{table}

\end{document}